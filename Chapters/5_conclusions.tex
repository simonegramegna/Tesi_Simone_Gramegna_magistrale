\chapter{Conclusions and future developments}
In this concluding chapter, I present some final thoughts and propose potential directions for future research concerning the development of PEO ontology.


\section{Conclusions}
The work described in this thesis aimed to create a resource representing two recently emerged technologies: large language models and prompt engineering. Given the novelty of these technologies, existing resources are fragmented and outdated, lacking the latest developments. This highlighted the need to create a comprehensive resource that thoroughly represents prompt engineering and large language models, along with related aspects such as architecture, solvable tasks, functionalities of large language models, and prompts that can be crafted using prompt engineering techniques. Starting the project with two research questions in mind: "Is it possible to formalize the knowledge related to large language models and prompting techniques?" and "Is it possible to use the model formalization to get additional knowledge?" I delved into state-of-the-art techniques in ontology engineering and ontology design patterns to understand the process to follow during development and the concepts that could be reused in the representation.After completing this study, I proceeded to examine the main state-of-the-art large language models, including both general-purpose and multimodal models. Finally, I delved into prompt engineering, studying and describing the leading state-of-the-art techniques in this field. This comprehensive study, which explored multiple aspects, provided me with a clear framework of what should be represented in the ontology and the steps to follow for its development. Following the LOT (Linked Open Terms) methodology, widely used in the industry for ontology and vocabulary development, I defined the project requirements. This included considering possible use cases, identifying sources for the information to be included in the ontology, outlining functional and non-functional requirements, and determining the ontology's purpose and scope. All this information is compiled in a document called the ORSD (Ontology Requirements Specification Document), which is valuable for ontology developers or anyone seeking to understand the ideas and problems addressed by the ontology. The work continued with the conceptualization of the ontology, taking into account ontology design patterns. This was followed by encoding the ontology in RDF and TTL formats, creating all necessary classes and relationships, and adequately populating it with instances of the various classes. Since manually populating the ontology is a lengthy and tedious task, I experimented with using GPT-4 to automatically populate the ontology, resulting in a second version of the PEO ontology. Both versions of the PEO ontology are tested and their quality and consistency were verified against the requirements defined at the outset. The evaluation provided a comprehensive overview of the quality of the results achieved. The ontology is consistent, includes a substantial number of classes, is not overly complex, and is capable of correctly addressing the competency questions, which were translated into SPARQL queries.

% da completare




\section{Future developments}

