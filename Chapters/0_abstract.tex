This thesis explores the development of a prompt engineering ontology to improve the understanding, organization, and application of large language models (LLMs) across various domains. By providing a structured representation of prompting techniques and their relationships with LLMs, the ontology serves as a valuable resource for guiding users in effectively interacting with these advanced systems.
Large Language Models (LLMs) are AI models based on large neural networks, trained on vast amounts of text to understand and generate natural language. Those models are interesting because they are revolutionizing human-machine communication by automate complex tasks like writing, translation and coding.
However, the currently available resources reviewing LLMs and prompt engineering lack and they are fragmented because both technologies are very recent, they are evolving quickly and there is no resource that  put them together.
We propose PEO, an ontology that models the domain of LLMs and prompt engineering in order to facilitate users in choosing the most appropriate techniques for solving a problem. PEO is designed to support a wide range of users, including researchers, developers, educators, and content creators. The research adopts a systematic approach to ontology development, following the LOT (Linked Open Terms) methodology.
We adopted a systematic approach to ontology development, following the Linked Open Terms (LOT) methodology and special attention is given to ontology evaluation and publication. We evaluate the ontology according to established experimental protocols.
The outcomes suggest that it is possible to formalize knowledge about large language models and prompt engineering, inferring new useful knowledge to the users.
PEO can provide support to a wide range of users, including researchers, developers, educators, and content creators.