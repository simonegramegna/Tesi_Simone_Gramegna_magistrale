\chapter{Introduction}
\section{Context}
Artificial intelligence (AI) has increasingly become an integral part of our life, having an undeniable impact on today’s society. 
AI was defined, for the first time, in 1955 at Darthmounth Research project as problem of \textit{"making a machine behave in ways that would be called intelligent if a human were so behaving"}\cite{kaplan2019siri}. 
Over the years, the focus has not been limited to the theoretical aspect alone. The rapid development of technologies, the increase in computing power, as well as the widespread presence of sensors have enabled the application of artificial intelligence as a support technology in fields ranging from industry, healthcare, and business to education. \cite{busnatu2022clinical}
Artificial intelligence is not only applied to these sectors but can also be found in common applications such as social media, digital assistants, recommendations, online searches, and facial recognition \cite{ref1}. These are just some of the applications we interact with in our daily lives.

\subsection{Large language models in real life}
The first voice assistants began to appear starting in 2011, when Apple introduced Siri on its new iPhone model: a voice assistant capable of conversing with users in natural language. Later, other assistants were introduced to the market, including Amazon Alexa (2014) and Google Assistant (2016) \cite{ref2}. These assistants generally have the same functionalities, being able to send messages and have simple conversations with users. However, starting in November 2022, the performance of these assistants has been surpassed by a new text-based assistant released by OpenAI: ChatGPT.\\
ChatGPT, in which GPT stands for \textit{Generative Pre-trained Transformer} a family of large language models created by OpenAI that uses deep learning to generate human-like, conversational text. 
ChatGPT represents a significant leap forward compared to the AI-based assistants available on the market until that time, as it can perform more advanced tasks than its competitors, including: writing a text/letter, coding, summarizing content, and writing Excel formulas, to name a few. Subsequent versions have integrated the DALL-E 3 model, capable of generating images, and have introduced GPT-4, the latest, more powerful, and up-to-date version of the large language model. \cite{ref3}

\subsection{Prompt engineering and the role of the prompt engineer}
The advent of advanced large language models like ChatGPT and BARD not only creates new opportunities for innovation and automation but also introduces significant challenges. One of the main challenges in using these models is creating and optimize prompts (questions posed to the model by the humans) \cite{ref5} that provide the model with the right instructions to generate accurate and relevant responses. Prompt engineering specifically addresses this challenge and focuses on defining the interactions and outputs of large language models, whose core purpose is to create optimal prompts for a generative model.\cite{amatriain2024prompt}
Everything lies in shaping the prompt while considering not only the user’s goal but also the context and the specific large language model being used, these aspects are considered by a new professional role that is emerging within companies: the prompt engineer. The prompt engineer must essentially select the most appropriate prompt engineering technique for a given task, a specific large language model, and the intended goal. The aim of this thesis is to provide a new tool to support their decisions: a prompt engineering ontology developed through an experimental approach.


\section{Thesis objective}
The knowledge related to prompt engineering techniques and large language models is available online in various heterogeneous and fragmented forms. There are several websites, blogs, repositories, and social media posts discussing prompt engineering and large language models. However, as of now, no resource exists that consolidates this knowledge in a clear, simple, and structured manner. Therefore, the objective of this thesis is to create a resource, an ontology that logically and systematically describes prompt engineering techniques and large language models: two concepts that, although they may seem different, are inherently interconnected. The use of an ontology allows for effectively expressing the connections between entities in these two domains by employing classes, instances, and relationships.
The ontology to be designed and implemented should be well-structured and comprehensive, including the most widely recognized prompt engineering techniques and state-of-the-art large language models. It must also be easily accessible, enabling users such as students, researchers, and developers to efficiently find the information they need.

\subsection{Thesis research questions}
Starting from the thesis objective:\\
\begin{center}
\textbf{\large Development of a prompt engineering and large language models ontology}
\end{center}

it is reasonable to ask questions whose answers will be provided at the end of the thesis.

\begin{itemize}
    \item \textbf{\large Question 1: 
Is it possible to formalize the knowledge related to large language models and prompting techniques?}

\item \textbf{\large Question 2: Is it possible to use the model formalization to get additional knowledge?}
\end{itemize}

Keeping in mind these two questions, I will develop the project and the thesis as follows.



\section{Thesis structure}
The subsequent chapters of this thesis are structured as follows. In Chapter 2, \textbf{"Background"}, I illustrate the theoretical foundations of the work, starting with the definition of ontology. I delve into the field of ontology engineering for the development of ontologies, exploring various ontology engineering methodologies and the main design patterns available in the state of the art. Subsequently, there is an in-depth analysis of state-of-the-art large language models, which will be represented in the ontology. The chapter concludes with a section dedicated to prompt engineering and the various prompt engineering techniques applicable to large language models.\\
In Chapter 3, \textbf{"Ontology design"}, the ontology design phase is described in detail, including an in-depth explanation of the various stages of the design process. In Chapter 4, \textbf{"Ontology implementation"}, starting from the artifacts produced during the design phase, the ontology is concretely conceptualized, taking into account design patterns, and implemented. The resulting ontology is evaluated using various state-of-the-art techniques and published online. The chapter concludes with a discussion of the results achieved. Finally , in the chapter 5, \textbf{"Conclusions and future developments"}, I conclude by summarizing the work accomplished and discussing potential future developments for the ontology based on the results achieved.